\documentclass[slidestop,mathserif,compress,xcolor=svgnames,table]{beamer} 
\mode<presentation>
{  
  \setbeamertemplate{background canvas}[vertical shading][bottom=blue!5,top=blue!5]
  \setbeamertemplate{navigation symbols}{}%{\insertsectionnavigationsymbol}
  \usetheme{LSU}
}

\usepackage{pgf,pgfarrows,pgfnodes,pgfautomata,pgfheaps,pgfshade}
\usepackage{amsmath,amssymb,amsfonts}
\usepackage{multirow}
\usepackage{tabularx}
\usepackage{booktabs}
\usepackage{colortbl}
\usepackage{tikz}
\usetikzlibrary{shapes,arrows}
\usetikzlibrary{calc}
\pgfdeclarelayer{background}
\pgfdeclarelayer{foreground}
\pgfsetlayers{background,main,foreground}
\usepackage[latin1]{inputenc}
\usepackage{colortbl}
\usepackage[english]{babel}
\usepackage{hyperref}
\usepackage{movie15}
\hypersetup{
  pdftitle={Introduction to High Performance Computing},
  pdfauthor={Alexander B. Pacheco, User Services Consultant, Louisiana State University}
}
%\usepackage{movie15}
\usepackage{times}

\setbeamercovered{dynamic}
\beamersetaveragebackground{DarkBlue!2}
\beamertemplateballitem

\usepackage[english]{babel}
\usepackage[latin1]{inputenc}
\usepackage[T1]{fontenc}
\usepackage{graphicx}
\definecolor{DarkGreen}{rgb}{0.0,0.3,0.0}
\definecolor{Blue}{rgb}{0.0,0.0,0.8} 
\definecolor{dodgerblue}{rgb}{0.1,0.1,1.0}
\definecolor{indigo}{rgb}{0.41,0.1,0.0}
\definecolor{seagreen}{rgb}{0.1,1.0,0.1}
\DeclareSymbolFont{extraup}{U}{zavm}{m}{n}
%\DeclareMathSymbol{\vardiamond}{\mathalpha}{extraup}{87}
\newcommand*\vardiamond{\textcolor{tigerspurple}{%
  \ensuremath{\blacklozenge}}}
\newcommand*\up{\textcolor{green}{%
  \ensuremath{\blacktriangle}}}
\newcommand*\down{\textcolor{red}{%
  \ensuremath{\blacktriangledown}}}
\newcommand*\const{\textcolor{darkgray}%
  {\textbf{--}}}


\setbeamercolor{uppercol}{fg=white,bg=red!30!black}%
\setbeamercolor{lowercol}{fg=black,bg=red!15!white}%
\setbeamercolor{uppercol1}{fg=white,bg=blue!30!black}%
\setbeamercolor{lowercol1}{fg=black,bg=blue!15!white}%%
\setbeamercolor{uppercol2}{fg=white,bg=green!30!black}%
\setbeamercolor{lowercol2}{fg=black,bg=green!15!white}%
\newenvironment{colorblock}[4]
{
\setbeamercolor{upperblock}{fg=#1,bg=#2}
\setbeamercolor{lowerblock}{fg=#3,bg=#4}
\begin{beamerboxesrounded}[upper=upperblock,lower=lowerblock,shadow=true]}
{\end{beamerboxesrounded}}
\newenvironment{ablock}[0]
{
\begin{beamerboxesrounded}[upper=uppercol,lower=lowercol,shadow=true]}
{\end{beamerboxesrounded}}
\newenvironment{bblock}[0]
{
\begin{beamerboxesrounded}[upper=uppercol1,lower=lowercol1,shadow=true]}
{\end{beamerboxesrounded}}
\newenvironment{eblock}[0]
{
\begin{beamerboxesrounded}[upper=uppercol2,lower=lowercol2,shadow=true]}
{\end{beamerboxesrounded}}


\title{Introduction to High Performance Computing}


\author[Alex Pacheco]{\large{Alexander~B.~Pacheco}}
       
%\institute[High Performance Computing @ Louisiana State University - http://www.hpc.lsu.edu] {\inst{}\footnotesize{User Services Consultant\\LSU HPC \& LONI\\sys-help@loni.org}}
\institute[HPC Training: Fall 2013] {\inst{}\footnotesize{User Services Consultant\\LSU HPC \& LONI\\sys-help@loni.org}}

\date[\hfill{September 4, 2013\hspace{2cm}\insertframenumber/\inserttotalframenumber}]{\scriptsize{HPC Training Fall 2013\\Louisiana State University\\Baton Rouge\\September 4, 2013}}
     
\subject{Talks}
\keywords{High Performance Computing, LONI \& LSU HPC Computing Resources}
% This is only inserted into the PDF information catalog. Can be left
% out. 




% If you have a file called "university-logo-filename.xxx", where xxx
% is a graphic format that can be processed by latex or pdflatex,
% resp., then you can add a logo as follows:

% Main Logo on bottom left
\pgfdeclareimage[height=0.55cm]{its-logo}{LONI}
\logo{\pgfuseimage{its-logo}}
% University Logo on top left
\pgfdeclareimage[height=0.55cm]{university-logo}{LSUGeauxPurp}
\tllogo{\pgfuseimage{university-logo}}
% Logo at top right
\pgfdeclareimage[height=0.5cm]{institute-logo}{its-logo}
\trlogo{\pgfuseimage{institute-logo}}
% Logo at bottom right
\pgfdeclareimage[height=0.5cm]{hpc-logo}{cct-logo}
\brlogo{\pgfuseimage{hpc-logo}}

% Delete this, if you do not want the table of contents to pop up at
% the beginning of each subsection:
 \AtBeginSection[]
 {
   \begin{frame}<beamer>
    \frametitle{\small{Outline}}
     \small
     \tableofcontents[currentsection,currentsubsection]
   \end{frame}
 }

\begin{document}

\frame{\titlepage}

\normalsize
\begin{frame}[label=toc,squeeze]
%  \footnotesize
  \frametitle{\small{Outline}}
  \tableofcontents
\end{frame}

%\part{Introduction}
\section{What is HPC - Background and Defintions}
\begin{frame}
  \frametitle{\small What is Computational Science?}
  \begin{itemize}
    \item Gain understanding, mainly through the analysis of mathematical models implemented on computers.
    \item Construct mathematical models and quantitative analysis techniques, using computers to analyze and solve scientific problems.
    \item Typically, these models require large amount of floating-point calculations not possible on desktops and laptops.
    \item The field's growth drove the need for HPC and benefited from it.
  \end{itemize}
\end{frame}

\begin{frame}
  \frametitle{\small What is HPC?}
  \begin{itemize}
    \item High Performance Computing (HPC) is computation at the forefront of modern technology, often done on a supercomputer. 
    \item Acronym soup:
    \begin{enumerate}
        \item HPC tasks are characterized as needing large amounts of computing power for short periods of time.
        \item High-throughput computing (HTC) tasks also require large amounts of computing, but for much longer times.
        \item High Productivity Computing Systems (HPCS) is a DARPA project for developing a new generation of economically viable high productivity computing systems.
    \end{enumerate}
  \end{itemize}
\end{frame}

\begin{frame}
  \frametitle{\small What is a supercomputer?}
  \begin{itemize}
    \item A supercomputer is a computer at the frontline of current processing capacity, particularly speed of calculation.
    \item A Supercomputer
    \begin{enumerate}
        \item in the 70's used only a few processors
        \item in the 90's machines with thousands of processors appeared
        \item currently, massively parallel supercomputers with tens of thousands of ``off-the-shelf'' processors were the norm.
    \end{enumerate}
    \item Today, commodity PC's which you can purchase off-the-self have more than one core i.e. dual core, quad core processors.
    \item Smartphones and Tablets today have more processing power than 15 year old Supercomputers.
    \item What was a Supercomputer 15 years ago now sits on your desk, or even in your hand.
 \end{itemize}
\end{frame}

\begin{frame}
  \frametitle{\small What is driving the Change?}
  \begin{columns}
    \column{0.45\textwidth}
    \begin{itemize}
      \item Moore's Law: number of transistors on integrated circuits doubles approximately every two years. 
      \item So how is Supercomputing performance measured and who ranks them?
     \end{itemize}
    \column{0.55\textwidth}
    \begin{center}
      \includegraphics[width=\textwidth]{Moores_Law2011}\\
		      {\tiny Source: \url{http://en.wikipedia.org/wiki/Moore's_law}}
    \end{center}
  \end{columns}
\end{frame}

\begin{frame}
  \frametitle{\small TOP 500}
  \begin{itemize}
    \item The TOP 500 project ranks and details the 500 most powerful known computer systems in the world, published semi-annually
    \item Performance is measured in \textbf{FL}oating point \textbf{O}perations \textbf{P}er \textbf{S}econd (FLOPS or flop/s)
    \item The most powerful supercomputers nowdays
    \begin{itemize}
      \item[-] have more than a million cores
      \item[-] operate in petaflops ($10^{15}$) range.
    \end{itemize}
    \item The fastest supercomputer as of June 2013 top500 list is Tianhe-2
    \begin{itemize}
      \item[-] Location: National University of Defense Technology
      \item[-] Nodes: 16000
      \item[-] Cores: 3120000
      \item[-] Peak Performance: 33.86 PFlop/s
    \end{itemize}
  \end{itemize}
  \begin{gather*}
    GFLOPs = cores \times clock \times \frac{FLOPs}{cycle}
  \end{gather*}
  \fontsize{4}{5}{\selectfont Most microprocessors today can do 4 FLOPs per clock cycle. Therefore a 2.5-GHz processor has a theoretical performance of 10 billion FLOPs = 10 GFLOPs.}
\end{frame}

\begin{frame}
  \frametitle{\small Top 500 Performance}
  \begin{center}
    \includegraphics[width=0.8\textwidth,clip=true]{Performance_Development}\\
    {\tiny Source: \url{http://www.top500.org/statistics/perfdevel/}}
  \end{center}
\end{frame}

\begin{frame}
  \frametitle{\small Top 500 Projected Performance}
  \begin{center}
    \includegraphics[width=0.8\textwidth,clip=true]{Projected_Performance_Development}\\
    {\tiny Source: \url{http://www.top500.org/statistics/perfdevel/}}
  \end{center}
\end{frame}

\begin{frame}
  \frametitle{\small Which fields are Supercomputers used in?}
  \begin{columns}
    \column{0.65\textwidth}
    \vspace{-0.5cm}
    \begin{center}
      \includegraphics[width=\textwidth,clip=true]{Top500-App-SystemShare}
    \end{center}
    \column{0.35\textwidth}
    \vspace{0.5cm}
    \begin{center}
      \hspace{-0.84cm}
      \includegraphics[width=1.2\textwidth,clip=true]{Top500-App-SystemShare1}\\
      {\tiny Source: \url{http://www.top500.org/statistics/overtime/}}
    \end{center}
  \end{columns}
\end{frame}

%\begin{frame}
%  \frametitle{\small }
%  \begin{center}
%    \includegraphics[width=0.6\textwidth,clip=true]{Top500-PerfDevel}\\
%    {\tiny Source: \url{http://www.top500.org/statistics/overtime/}}
%  \end{center}
%\end{frame}

\begin{frame}
  \frametitle{\small How many cores per socket are used?}
  \vspace{-0.5cm}
  \begin{center}
    \includegraphics[width=0.6\textwidth,clip=true]{Top500_Cores-SystemShare}\\
    {\tiny Source: \url{http://www.top500.org/statistics/overtime/}}
  \end{center}
\end{frame}

\begin{frame}
  \frametitle{\small Types of Processors: Acclerators, GPUs etc}
  \begin{columns}
    \column{0.65\textwidth}
    \vspace{-0.5cm}
    \begin{center}
      \includegraphics[width=\textwidth,clip=true]{Top500_Acc-SystemShare}
    \end{center}
    \column{0.35\textwidth}
    \vspace{0.5cm}
    \begin{center}
      \hspace{-1cm}
      \includegraphics[width=\textwidth,clip=true]{Top500_Acc-SystemShare1}\\
      {\tiny Source: \url{http://www.top500.org/statistics/overtime/}}
    \end{center}
  \end{columns}
\end{frame}


\begin{frame}
  \frametitle{\small Which Countries have Supercomputers?}
  \vspace{-0.5cm}
  \begin{center}
    \includegraphics[width=0.6\textwidth,clip=true]{Top500_Country-SystemShare}\\
    {\tiny Source: \url{http://www.top500.org/statistics/overtime/}}
  \end{center}
\end{frame}

\begin{frame}
  \frametitle{\small Which OS are used in Supercomputers?}
  \begin{columns}
    \column{0.65\textwidth}
    \vspace{-1cm}
    \begin{center}
      \includegraphics[width=0.9\textwidth,clip=true]{Top500_OS-SystemShare}
    \end{center}
    \column{0.35\textwidth}
    \vspace{0.5cm}
    \begin{center}
      \hspace{-1cm}
      \includegraphics[width=1.2\textwidth,clip=true]{Top500_OS-SystemShare1}\\
      {\tiny Source: \url{http://www.top500.org/statistics/overtime/}}
    \end{center}
  \end{columns}
\end{frame}

\begin{frame}
  \frametitle{\small Which OS are used in Supercomputers?}
  \vspace{-0.5cm}
  \begin{center}
    \includegraphics[width=0.6\textwidth,clip=true]{Top500_OSFam-SystemShare}\\
    {\tiny Source: \url{http://www.top500.org/statistics/overtime/}}
  \end{center}
\end{frame}

\begin{frame}
  \frametitle{\small Who uses these monstrous systems?}
  \vspace{-0.5cm}
  \begin{center}
    \includegraphics[width=0.6\textwidth,clip=true]{Top500_Segments-SystemShare}\\
    {\tiny Source: \url{http://www.top500.org/statistics/overtime/}}
  \end{center}
\end{frame}

\begin{frame}
  \frametitle{\small Why use HPC?}
  \begin{itemize}
    \item HPC may be the only way to achieve computational goals in a given amount of time
    \begin{itemize}
      \item Size: Many problems that are interesting to scientists and engineers cannot fit on a PC usually because they need more than a few GB of RAM, or more than a few hundred GB of disk.
      \item Speed: Many problems that are interesting to scientists and engineers would take a very long time to run on a PC: months or even years; but a problem that would take a month on a PC might only take a few hours on a supercomputer
    \end{itemize}
  \end{itemize}
\end{frame}

\begin{frame}
  \frametitle{\small Who uses HPC? Parallel Computing}
  \begin{itemize}
    \item many calculations are carried out simultaneously
    \item based on principle that large problems can often be divided into smaller ones, which are then solved in parallel
    \item Parallel computers can be roughly classified according to the level at which the hardware supports parallelism.
    \begin{enumerate}
      \item Multicore computing
      \item Symmetric multiprocessing
      \item Distributed computing
      \item Grid computing
      \item General-purpose computing on graphics processing units (GPGPU)
    \end{enumerate}
  \end{itemize}
\end{frame}

\begin{frame}
  \frametitle{\small Distributed Computing}
  \begin{itemize}
    \item using resources to solve a problem by dividing into many tasks, each of which is solved by one or more computers connected by a network.
    \item Condor
    \begin{itemize}
      {\footnotesize
      \item[-] Work: run my application
      \item[-] Resources: a dedicated cluster of computers or idle computers on a university network
      }
    \end{itemize}
    \item Google
    \begin{itemize}
      {\footnotesize
      \item[-] Work: process a query or "google search''
      \item[-] Resources: a lot of servers located worldwide
      }
    \end{itemize}
    \item SETI@HOME
    \begin{itemize}
      {\footnotesize
      \item[-] Work: process signal data to find ET
      \item[-] Resources: a few servers and lots of PCs worldwide
      }
    \end{itemize}
    \item FOLDING@HOME
    \begin{itemize}
      {\footnotesize
      \item[-] Work: simulates protein folding, computational drug design, and other types of molecular dynamics.
      \item[-] Resources: idle processing resources of thousands of PCs of volunteer who have installed the software on their systems.
      }
    \end{itemize}
  \end{itemize}
\end{frame}

\begin{frame}
  \frametitle{\small Grid Computing}
  \begin{itemize}
    \item A subset of Distributed Computing
    \item Three point checklist by Ian Foster\let\thefootnote\relax\footnote{\tiny I. Foster. \textit{"What is the Grid? A Three Point Checklist"}, \url{http://dlib.cs.odu.edu/WhatIsTheGrid.pdf}}
    \begin{enumerate}
      \item coordinates resources that are not subject to centralized control,
      \item using standard, open, general-purpose protocols and interfaces,
      \item to deliver nontrivial qualities of service.
    \end{enumerate}
    \item Condor across a university fails test 2
    \item Google has issues on tests 1 and 2
    \item SETI@HOME passes all three tests
  \end{itemize}
\end{frame}

\begin{frame}
  \frametitle{\small Volunteer Computing}
  \begin{itemize}
    \item Volunteer computing is a type of distributed computing in which computer owners donate their computing resources (such as processing power and storage) to one or more ``projects''.
    \item \textbf{B}erkeley \textbf{O}pen \textbf{I}nfrastructure for \textbf{N}etwork \textbf{C}omputing (BOINC) is an example platform that supports over 40 projects (as of Dec 2012 $\sim$ 7.279 PFLOPS \let\thefootnote\relax\footnote{\tiny Source: Wikipedia})
    \begin{enumerate}
      \item SETI@Home (730 TFLOPS)
      \item MilkyWay@Home (1.6 PFLOPS)
      \item Einstein@Home (210 TFLOPS)
    \end{enumerate}
    \item Folding@Home is another example with 4.195 PFLOPS as of Dec 2009
  \end{itemize}
\end{frame}

\begin{frame}
  \frametitle{\small What does HPC do?}
  \begin{columns}
    \column{0.65\textwidth}
    \vspace{-1cm}
    \begin{itemize}
      \item Simulation of Physical Phenomena
      \begin{itemize}
        \item Storm Surge Prediction
        \item Black Holes Colliding
        \item Molecular Dynamics
      \end{itemize}
      \item Data analysis and Mining
      \begin{itemize}
        \item Bioinformatics
        \item Signal Processing
        \item Fraud detection
      \end{itemize}
      \item Visualization
      \item Design
      \begin{itemize}
        \item Supersonic ballute
        \item Boeing 787 design
        \item Drug Discovery
        \item Oil Exploration and Production
        \item Automotive Design
        \item Art and Entertainment
      \end{itemize}
    \end{itemize}
    \column{0.35\textwidth}
    \includegraphics[width=0.6\textwidth,clip=true]{./Isaac-Storm-Surge}\\
    \includegraphics[width=0.6\textwidth,clip=true]{./Colliding-Black-Holes}\\
    \includegraphics[width=0.6\textwidth,clip=true]{./Molecular-Dynamics}\\
    \includegraphics[width=0.6\textwidth,clip=true]{./Plane-Design}
  \end{columns}
\end{frame}

\begin{frame}
  \frametitle{\small HPC by Disciplines}
  \begin{itemize}
    \item Traditional Disciplines
    \begin{itemize}
      \item Science: Physics, Chemistry, Biology, Material Science
      \item Engineering
    \end{itemize}
    \item Non Traditional Disciplines
    \begin{itemize}
      \item Finance
      \begin{itemize}
        \item Preditive Analytics
        \item Trading
      \end{itemize}
      \item Humanities
      \begin{itemize}
        \item Culturomics or cultural analytics: study human behavior and cultural trends through quantitative analysis of digitized texts, images and videos.
      \end{itemize}
    \end{itemize}
  \end{itemize}
\end{frame}

\section{Available HPC Resources}
\begin{frame}
  \frametitle{\small Which HPC resources are available?}
  \begin{columns}
    \column{0.6\textwidth}
    \begin{itemize}
      \item National Level: E\textbf{x}treme \textbf{S}cience and \textbf{E}ngineering \textbf{D}iscovery \textbf{E}nvironment (xSEDE)
      \item[-] 5 year, \$121M project supported by NSF
      \item[-] supports 16 supercomputers and high-end visualization and data analysis resources across the country.
    \end{itemize}
    \column{0.4\textwidth}
    \begin{center}
      \includegraphics[width=\textwidth]{xsede-logo}
    \end{center}
  \end{columns}
\end{frame}
\begin{frame}
  \frametitle{\small Which HPC resources are available?}
  \begin{columns}
    \column{0.6\textwidth}
    \begin{itemize}
      \item State Level: \textbf{L}ouisiana \textbf{O}ptical \textbf{N}etwork \textbf{I}nitiative (LONI)
      \item[-] A state-of-the-art fiber optic network that runs throughout Louisiana and connects Louisiana and Mississippi research universities.
      \item[-] \$40M Optical Network, 10Gb Ethernet over fiber optics.
      \item[-] \$10M Supercomputers installed at 6 sites.
    \end{itemize}
    \column{0.4\textwidth}
    \begin{center}
      \includegraphics[width=\textwidth]{LONI}
    \end{center}
  \end{columns}
\end{frame}
\begin{frame}
  \frametitle{\small Which HPC resources are available?}
  \begin{columns}
    \column{0.5\textwidth}
    \begin{itemize}
      \item University Level: LSU HPC resources available to LSU Faculty and their affiliates.
      \item[$\bigstar$] LONI and LSU HPC administered and supported by HPC@LSU
    \end{itemize}
    \column{0.5\textwidth}
    \begin{center}
      \includegraphics[width=\textwidth]{hpc-logo}\\
      \vspace{0.5cm}
      \includegraphics[width=\textwidth]{its-logo}
    \end{center}
  \end{columns}
\end{frame}


\section{What is HPC@LSU?}
\begin{frame}
  \frametitle{\small What is HPC@LSU?}
  \begin{itemize}
    \item Hardware Resources:
    \begin{itemize}
      \item Currently manage 11 computing resources.
      \item 7 LONI computing clusters
      \item 4 LSU HPC computing clusters
    \end{itemize}
    \item Available Software Stack
    \begin{itemize}
      \item Communication Software
      \item Programming support: Compilers and Libraries
      \item Application Software
    \end{itemize}
    \item User Services
    \begin{itemize}
      \item Support: running jobs, software installation 
      \item Training: 
    \end{itemize}
  \end{itemize}
\end{frame}

\begin{frame}
  \frametitle{\small LONI \& LSU HPC Clusters}
  \begin{columns}
    \column{11cm}
    \vspace{-0.7cm}
  \footnotesize{
    \begin{block}{Linux Clusters}
      \begin{center}
        \begin{tabular}{|c|c|c|c|c|c|}
          \hline
          Resource & Cluster & Peak TF/s & Location & Status & Login\\
          \hline
          \multirow{6}{*}{LONI} & QueenBee & 50.7 & ISB & Production & LONI \\
          & Eric & 4.7 & LSU & Production & LONI\\
          & Louie & 4.7 & Tulane & Production & LONI\\
          & Oliver & 4.7 & ULL & Production & LONI\\
          & Painter & 4.7 & LaTech & Production & LONI\\
          & Poseidon & 4.7 & UNO & Production & LONI\\
          & Satellite & 4.7 & Southern & Unknown & LONI\\
          \hline
          \multirow{4}{*}{ LSU HPC} & Tezpur & 15.3 & LSU & Production & HPC\\
          & Philip & 3.5 & LSU & Production & HPC\\
          \cline{2-6}
          & \multirow{2}{*}{SuperMike II} & 146 (CPU) & \multirow{2}{*}{LSU} & \multirow{2}{*}{Production} & \multirow{2}{*}{HPC} \\
          &                            & 66 (GPU) & &  & \\
          \hline
        \end{tabular}
      \end{center}
    \end{block}
    \begin{block}{AIX Clusters}
      \begin{center}
        \def\firstrowcolor{\rowcolor{green}}
        \def\secondrowcolor{\rowcolor{blue!50}}
        \def\thirdrowcolor{\rowcolor{tigerspurple!80}}
        \begin{tabular}{|c|c|c|c|c|c|}
          \hline
          Resource & Cluster & Peak TF/s & Location & Status & Login\\
          \hline
          LSU HPC & Pandora & 6.8 & LSU & Production & HPC\\
          \hline
        \end{tabular}
      \end{center}
    \end{block}
  }
  \end{columns}
\end{frame}

\begin{frame}[allowframebreaks]
  \frametitle{\small Cluster Hardware}
  \vspace{-0.3cm}
  \begin{itemize}
  \item Queen Bee 
    \begin{enumerate}
      {\footnotesize
      \item[$\vardiamond$]668 nodes: 8 Intel Xeon cores @ 2.33 GHz
      \item[$\vardiamond$]8 GB RAM
      \item[$\vardiamond$]192 TB storage
      }
    \end{enumerate}
  \item Other LONI clusters
    \begin{enumerate}
      {\footnotesize
      \item[$\vardiamond$]128 nodes: 4 Intel Xeons cores @ 2.33 GHz
      \item[$\vardiamond$]4 GB RAM
      \item[$\vardiamond$]9 TB storage
      }
    \end{enumerate}
  \item Tezpur
    \begin{enumerate}
      {\footnotesize
      \item[$\vardiamond$]360 nodes, 4 Intel Xeon cores @ 2.33 GHz
      \item[$\vardiamond$]4 GB RAM
      \item[$\vardiamond$]32 TB storage
      }
    \end{enumerate}
  \item Pandora
    \begin{enumerate}
      {\footnotesize
      \item[$\vardiamond$]8 Power7 nodes, 8 IBM Power7 processors @ 3.33 GHz
      \item[$\vardiamond$]128 GB RAM
      \item[$\vardiamond$]19 TB storage
      }
    \end{enumerate}
    \newpage
  \item Philip
    \begin{enumerate}
      {\footnotesize
      \item[$\vardiamond$]37 nodes, 8 Intel Xeon cores @ 2.93 GHz
      \item[$\vardiamond$]24/48/96 GB RAM
      \item[$\vardiamond$]2 nodes, 12 Intel Xeon core @ 2.66GHz with hyperthreading with 3 Tesla 2070 GPU's each
      \item[$\vardiamond$] Tesla M2070: 448 CUDA cores @ 1.15GHz and 5.25GB Total Memory
      \item[$\vardiamond$]Shares storage with Tezpur
      }
    \end{enumerate}
  \end{itemize}
\end{frame}

\begin{frame}[c]
  \frametitle{\small SuperMike II}
  \begin{eblock}{}
    \begin{itemize}
      \item Ranked 250 in Nov 2012 Top 500 List.
      \item 146 CPU TFlops and 66 double-precision GPU TFlops,
      \item 440 nodes, dual 8-core  Intel Sandybridge Xeon cores @ 2.6 GHz
      \item 382 standard nodes with 32GB RAM (16 cores per node),
      \item 50 GPU nodes with 64GB RAM and dual NVIDIA Tesla M2090 6GB GPUs,
      \item 8 big memory nodes with 256GB RAM, capable of aggregation into a single virtual symmetric processing (vSMP) node using ScaleMP,
      \item Mellanox Infiniband QDR network of 2:1 over-subscription.
%      \item Estimated production launch Feb 2013.
%      \item {\color{red} To run jobs on SuperMike II you will need}
%      \begin{enumerate}
%        {\color{red}
%          \item a LSU HPC account, and
%          \item an active LSU HPC allocation.
%        }
%      \end{enumerate}
    \end{itemize}
  \end{eblock}
\end{frame}

\begin{frame}
  \frametitle{\small QueenBee}
  \includegraphics[width=\textwidth]{QueenBee-02.jpg}
\end{frame}
\begin{frame}
  \frametitle{\small Painter Wiring}
  \includegraphics[width=\textwidth]{Painter.jpg}
\end{frame}
\begin{frame}
  \frametitle{\small Storage}
  \includegraphics[width=\textwidth]{Painter-01.jpg}
\end{frame}
\begin{frame}
  \frametitle{\small SuperMike II}
  \includegraphics[width=\textwidth]{supermike2-1.jpg}
\end{frame}

\begin{frame}
  \frametitle{\small Account Eligibility}
  \begin{bblock}{LONI}
    {\fontsize{9}{10}\selectfont
      \begin{itemize}
        \item All faculty and research staff at a LONI Member Institution, as well as students pursuing sponsored research activities at these facilities, are eligible for a LONI account. 
        \item Requests for accounts by research associates not affiliated with a LONI Member Institution will be handled on a case by case basis. 
        \item For prospective LONI Users from a non-LONI Member Institution, you are required to have a faculty or research staff in one of LONI Member Institutions as your Collaborator to sponsor you a LONI account.
      \end{itemize}
    }
  \end{bblock}
  \begin{bblock}{LSU HPC}
    {\fontsize{9}{10}\selectfont
      \begin{itemize}
        \item All faculty and research staff at Louisiana State University, as well as students pursuing sponsored research activities at LSU, are eligible for a LSU HPC account. 
        \item For prospective LSU HPC Users from outside LSU, you are required to have a faculty or research staff at LSU as your Collaborator to sponsor you a LSU HPC account.
      \end{itemize}
    }
  \end{bblock}
\end{frame}

\begin{frame}
  \frametitle{\small How do I get a LONI Account}
  \begin{bblock}{LONI Account}
    \begin{enumerate}
      \item Visit \url{https://allocations.loni.org/login_request.php}.
      \item Enter your \textbf{Institutional Email Address} and captcha code.
      \item Check your email and click on the link provided (link is active for 24hrs only)
      \item Fill the form provided
      \item For LONI Contact/Collaborator field enter the name of your research advisor/supervisor who must be a Full Time Faculty member at a LONI member institution.
      \item Click Submit button
      \item Your account will be activated once we have verified your credentials.
    \end{enumerate}
  \end{bblock}
\end{frame}

\begin{frame}
  \frametitle{\small How do I get a LSU HPC Account}
  \begin{bblock}{LSU HPC Account}
    \begin{enumerate}
      \item Visit \url{https://accounts.hpc.lsu.edu/login_request.php}.
      \item Enter your \textbf{Institutional Email Address} and captcha code.
      \item Check your email and click on the link provided (link is active for 24hrs only)
      \item Fill the form provided
      \item For HPC Contact/Collaborator field enter the name of your research advisor/supervisor who must be a Full Time Faculty member at LSU
      \item Click Submit button
      \item Your account will be activated once we have verified your credentials.
    \end{enumerate}
  \end{bblock}
\end{frame}

\begin{frame}
  \frametitle{\small LONI \& LSU HPC Accounts}
  \begin{ablock}{}
    \begin{itemize}
      {\huge
          \item LSU HPC and LONI systems are two distinct computational resources administered by HPC@LSU.
          \item[]
          \item Having an account on one does not grant the user access to the other.
      }
    \end{itemize}
  \end{ablock}
\end{frame}

\begin{frame}
  \frametitle{\small Allocations}
  \begin{bblock}{What is an Allocation?}
    {\footnotesize
      \begin{itemize}
        \item An allocation is a block of computer time measured in core-hours (number of processing cores requested times the amount of wall-clock time used in hours).
        \item LONI users: All jobs need to be charged to valid allocation.
        \item \color{red!90!black}{LSU HPC users: Allocations are coming soon when SuperMike II is put into production.}
      \end{itemize}
    }
  \end{bblock}
  \begin{bblock}{Who can request an Allocation?}
    {\footnotesize
      \begin{itemize}
        \item Only Full Time Faculty member at LONI member institutions can act as Principle Investigators (PI) and request Allocations.
        \item Rule of Thumb: If you can sponsor user accounts, you can request allocations.
        \item Everyone else will need to join an existing allocation of a PI usually your advisor/supervision or course instructor (if your course requires a LONI account). 
      \end{itemize}
    }
  \end{bblock}
\end{frame}

\begin{frame}
  \frametitle{\small How to request/join an Allocation}
  \begin{bblock}{}
    {\footnotesize
      \begin{itemize}
        \item Login to your LONI Profile at \url{https://allocations.loni.org}
        \item Click on "Request Allocation" in the right sidebar.
        \item Click "New Allocation" to request a New Allocation.
        \begin{enumerate}
          {\footnotesize
            \item Fill out the form provided.
            \item All requests require submission of a proposal justifying the use of the resources.
            \item Click "Submit Request" button.
          }
        \end{enumerate}
        \item Click "Join Allocation" to join an existing Allocation.
        \begin{enumerate}
          {\footnotesize
            \item Search for PI using his/her email address, full name or LONI username
            \item Click "Join Projects" button associated with the PI's information.
            \item You will be presented with a list of allocations associated with the PI. Click "Join" for the allocation you wish to join.
            \item Your PI will receive an email requesting him to confirm adding you to the allocation.
            \item Please do not contact the helpdesk to do this.
          }
        \end{enumerate}
      \end{itemize}
    }
  \end{bblock}
\end{frame}

\begin{frame}
  \frametitle{\small More on LONI Allocation}
  \begin{bblock}{Allocation Types}
    \begin{enumerate}
      \item \textbf{Startup}: Allocations upto 50K SUs
      \begin{itemize}
        \item Can be requested at any time during the year.
        \item Reviewed and Approved by the LONI Resource Allocation Committee.
        \item Only \textbf{two active} allocations per PI at any time.
        \item Expired Allocations are considered active if the end date is in the future.
      \end{itemize}
      \item \textbf{Large}: Allocations between 50K - 4M SUs.
      \begin{itemize}
        \item Reviewed and Approved by the LONI Resource Allocation Committee every Quater.
        \item Users can have multiple Large Allocations.
        \item A PI may have a total of 6M SUs active at any given time.
      \end{itemize}
    \end{enumerate}
  \end{bblock}
\end{frame}

\begin{frame}
  \frametitle{\small Account Management}
  \begin{bblock}{}
    \begin{itemize}
    \item LONI account
    \item[] \url{https://allocations.loni.org}
    \item LSU HPC account
    \item[] \url{https://accounts.hpc.lsu.edu}
    \item Newest cluster in production at LSU HPC is Pandora.
    \item Newest cluster at LSU HPC is SuperMike II is in user friendly mode. 
    \end{itemize}
  \end{bblock}
  
  \begin{eblock}{}
    \begin{itemize}
    \item The default Login shell is bash
    \item Supported Shells: bash, tcsh, ksh, csh \& sh
    \item Change Login Shell at the profile page
    \end{itemize}
  \end{eblock}
\end{frame}

\begin{frame}
  \frametitle{\small How do I reset my password?}
  \begin{itemize}
    \item LONI: Visit \url{https://allocations.loni.org/user_reset.php}
    \item LSU HPC: Visit \url{https://accounts.hpc.lsu.edu/user_reset.php}
    \item Enter the email address attached to your account and captcha code
    \item You will receive an email with link to reset your password, link must be used within 24 hours.
    \item Once you have entered your password, one of the HPC Admins need to approve the password reset.
    \item The Password approval can take anything from 10 mins to a few hours depending on the schedule of the Admins and also time of day
    \item You will receive a confirmation email stating that your password reset has been approved.
  \end{itemize}
\end{frame}

\begin{frame}
  \frametitle{\small Password Security}
  \begin{itemize}
    \item Passwords should be changed as soon as your account is activated for added security.
    \item Password must be at least 12 and at most 32 characters long, must contain three of the four classes of characters:
    \begin{enumerate}
      \item lowercase letters,
      \item uppercase letters, 
      \item digits, and 
      \item other special characters (punctuation, spaces, et cetera).
    \end{enumerate}
    \item Do not use a word or phrase from a dictionary,
    \item Do not use a word that can be obviously tied to the user which are less likely to be compromised.
    \item Changing the password on a regular basis also helps to maintain security.
    \item {\scriptsize \url{http://www.thegeekstuff.com/2008/06/the-ultimate-guide-for-creating-strong-passwords/}}
    \item {\scriptsize \url{http://en.wikipedia.org/wiki/Password_policy}}
  \end{itemize}
\end{frame}

\section{HPC@LSU Services}
\begin{frame}[c]
  \frametitle{\small Communication Software}
  \begin{itemize}
    \item Shared Memory Programming - OpenMP
    \begin{itemize}
      \item[$\vardiamond$] Good for programs that exhibit \textit{data parallelism}.
      \item[$\vardiamond$] Managed by compiler via special programming statements.
    \end{itemize}
    \item Distributed Memory Programming - MPI
    \begin{itemize}
      \item[$\vardiamond$] Good for programs that exhibit \textit{task parallelism}.
      \item[$\vardiamond$] Managed by programmer with library function calls.
    \end{itemize}
    \item Hybrid Programming - OpenMP + MPI
  \end{itemize}
\end{frame}

\begin{frame}[c]
  \frametitle{\small Programming Support}
  \begin{itemize}
    \item Compilers
    \begin{itemize}
      \item[$\vardiamond$] Intel Fortran and C/C++
      \item[$\vardiamond$] GNU compiler suite
      \item[$\vardiamond$] Portland group Fortran and C/C++
      \item[$\vardiamond$] CUDA 
    \end{itemize}
    \item Scripting languages
      \begin{enumerate}
        \item[$\vardiamond$] Perl, Python, BASH, TCSH, TCL/TK
      \end{enumerate}
%    \item Scientific and utility libraries
    \item Numerical and utility libraries
      \begin{enumerate}
        \item[$\vardiamond$] FFTW, HDF5, NetCDF, PetSc, Intel MKL
      \end{enumerate}
%    \item Debugging and Profiling Tools: Totalview, DDT, TAU
    \item Debugging and Profiling Tools
      \begin{enumerate}
        \item[$\vardiamond$] DDT, TAU, TotalView
      \end{enumerate}
  \end{itemize}
\end{frame}

\begin{frame}[c]
  \frametitle{\small Scientific Libraries}
  \begin{itemize}
    \item Some things are common in scientific codes
    \begin{itemize}
      \item Experts have developed and optimized methods for things like
      \begin{enumerate}
        \item Matrix Operations
        \item Fast Fourier Transform
      \end{enumerate}
      \item You do not need to reinvent the wheel, makes use of work done by the experts
    \end{itemize}
    \item Many Scientific Libraries are available
    \begin{enumerate}
      \item Linear Algebra: BLAS, ATLAS
      \item Linear Solvers: Scalapack, SuperLU, HYPRE, Intel MKL
      \item Fast Fourier Transform: FFTW, Intel MKL
      \item Boost
    \end{enumerate}
  \end{itemize}
\end{frame}

\begin{frame}[c]
  \frametitle{\small I/O}
  \begin{itemize}
    \item Reading and Writing data is another common problem
    \begin{itemize}
      \item ASCII: portable but slow
      \item Binary: fast but not portable across machine architectures
      \item what about metadata
    \end{itemize}
    \item Common HPC I/O libraries
    \begin{itemize}
      \item HDF: Hierarchial Data Format
      \item NetCDF: Network Common Data Format
      \item Manage format conversions between machines, can be annotated by metadata.
      \item Used by many application, such as third-party visualization applications.
    \end{itemize}
  \end{itemize}
\end{frame}

\begin{frame}[c]
  \frametitle{\small Programming? Why do I care?}
  \begin{itemize}
    \item Get your hands dirty:
    \begin{itemize}
      \item Roll your own code
      \item Install a source code application release
      \item Modify an existing code
      \item Understand what the code does for and to you
    \end{itemize}
    \item ELSE, use an existing or installed packages and hope it satisfies all your research needs.
  \end{itemize}
\end{frame}

\begin{frame}[c]
  \frametitle{\small Application Software}
  \begin{itemize}
    \item Quantum Chemistry
      \begin{enumerate}
        \item[$\blacksquare$] Gaussian, GAMESS, NWCHEM, CPMD
      \end{enumerate}
    \item Molecular Dynamics
      \begin{enumerate}
        \item[$\blacksquare$] Amber, Gromacs, LAMMPS, NAMD
      \end{enumerate}
    \item Engineering
      \begin{enumerate}
        \item[$\blacksquare$] Fluent (LSU only)
      \end{enumerate}
    \item Mathematics and Statistics
      \begin{enumerate}
        \item[$\blacksquare$] Matlab (LSU only), Mathematica (LSU only), Octave, R
      \end{enumerate}
    \item Visualization
      \begin{enumerate}
        \item[$\blacksquare$] GaussView, VisIt, VMD, GNUPLOT
      \end{enumerate}
  \end{itemize}
\end{frame}


\begin{frame}
  \frametitle{\small User Services}
  \begin{itemize}
    \item Consulting Services
    \begin{enumerate}
      \item[$\vardiamond$] Usage Problems, Program Optimization, Software Installation, Software development advice. 
    \end{enumerate}
    \item User Guides
    \begin{enumerate}
      \item[$\vardiamond$]HPC: \url{http://www.hpc.lsu.edu/docs/guides.php\#hpc}
      \item[$\vardiamond$]LONI: \url{http://www.hpc.lsu.edu/docs/guides.php\#loni}
    \end{enumerate}
  \item Documentation: \url{https://docs.loni.org}
  \item Online Courses: \url{https://docs.loni.org/moodle}
  \item Contact us
    \begin{enumerate}
      {\footnotesize
      \item[$\vardiamond$]Email ticket system: {\color{tigersblue}\href{mailto:sys-help@loni.org?Subject=Questions About HPC@LSU}{sys-help@loni.org}}
      \item[$\vardiamond$]Telephone Help Desk: 225-578-0900
      \item[$\vardiamond$]Instant Messenger: lsuhpchelp (AIM, Yahoo Messenger, Google Talk)
      }
    \end{enumerate}
  \end{itemize}
\end{frame}

\begin{frame}
  \frametitle{\small Educational Activities}
  \begin{itemize}
    \item Weekly Trainings
    \begin{enumerate}
      \item Introductory: User Environment, Linux
      \item Programming: Shell Scripting, Perl, Python, MPI, OpenMP
      \item Software Development: Debugging, Profiling, Make, Subversion
      \item Software Applications: Molecular Dynamics, Computational Chemistry \& Biology, Octave, MatLab
    \end{enumerate}
    \item Workshops
    \begin{enumerate}
      \item Programming: Fortran, C, C++
      \item Parallel Programming: MPI, OpenMP, GPU, OpenACC
      \item Support Workshops organized through other Departments and Supercomputing centers.
    \end{enumerate}
    \item User Symposium.
    \item[$\vardiamond$] Held in June 2012 \& 2013, Researchers from various LONI institutions presented their research via invited talks and poster sessions. 
  \end{itemize}
\end{frame}

\section{Wrap-Up}
\begin{frame}
  \frametitle{\small US Council on Competitiveness}
  \begin{itemize}
    \item HPC initiative  is intended to stimulate and facilitate wider usage of HPC across the private sector to propel productivity, innovation and competitiveness.\let\thefootnote\relax\footnote{\tiny \url{http://www.compete.org/about-us/initiatives/hpc}}
    \item Goals
    \begin{enumerate}
      \item Analyze the economic rationale for sustaining U.S. leadership in HPC, especially the impact upon manufacturing, services, business, and state-of-the-art research capabilities
      \item Identify key private sector HPC applications needs and priorities
      \item Identify workforce education and training needs to integrate HPC in the private sector
      \item Foster public-private sector partnerships to better leverage resources and expertise to help overcome barriers to more widespread private sector usage
    \end{enumerate}
  \end{itemize}
\end{frame}

\begin{frame}
  \frametitle{\small The Future is now}
  {\Large
    \begin{itemize}
      \item[] The FUTURE is now.
      \item[] {\color{blue}Whatever happens in supercomputing now will be in your desktop in 10-15 years.}
      \item[] Having supercomputing experience now will keep you ahead of the curve when things get to the desktop in a decade or two.
    \end{itemize}
  }
\end{frame}

\begin{frame}
  \frametitle{}
  \color{DarkBlue}{
  \begin{center}
    {\fontsize{40}{60}\selectfont The End}\\
    \vspace{1cm}
    {\fontsize{20}{30}\selectfont Any Questions?}\\
    \vspace{0.5cm}
    {\fontsize{15}{30}\selectfont Next Week:\\\color{red!80!black}{Introduction to Linux}}\\
    \vspace{0.5cm}
    {\fontsize{15}{20}\selectfont Survey: \color{red!90!white}{\url{http://www.hpc.lsu.edu/survey}}}
  \end{center}
  }
\end{frame}
\end{document}

